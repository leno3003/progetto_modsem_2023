\documentclass[12pt]{book}
\usepackage[italian]{babel}
\usepackage{amsfonts}
\usepackage{amsmath}
\usepackage{tcolorbox}
\title{Modellazione Concettuale per il Web Semantico \\ A.A. 2023-2024}
\author{Enrico Cassano \\ N.M. 912344}
\date{}

\usepackage{hyperref}

\begin{document}

\maketitle

\tableofcontents

\chapter{Motivazioni}

Gli scacchi sono un gioco nato in India nel VI secolo D.C, e sono
giunti in Europa attorno all'anno 1000, grazie agli influssi arabi
presenti nella penisola iberica. Raggiunsero la forma quasi attuale nel XV
secolo in Italia e in Spagna, ed il regolamento corrente fu definito
attorno al 1880. È uno dei giochi da tavolo più popolari al mondo,
grazie alla possibilità di essere giocato pressoché ovunque. Permette,
inoltre, di essere interpretato in modo agonistico,
grazie alla presenza di un organo regolatore, la FIDE.

Fin dal Medioevo, il gioco degli scacchi ha ricoperto un ruolo sociale
e culturale nelle corti e nella nobiltà di tutta Europa. Spesso, infatti,
venivano usati come metafora per la guerra.


\begin{figure}[h]
  \caption{The Chess Players, Honore Daumier}
  \centering
  \label{fig:chessplayers}
  \includegraphics[scale=1]{~/Pictures/Honoré_Daumier_032.jpg}
\end{figure} 

Anche nell'ambito artistico, gli scacchi sono stati fonte di
ispirazione per la realizzazione di diverse opere, fra cui dipinti
come \textit{The Chess Players} di Honore Daumier (figura
\ref{fig:chessplayers}). Sono stati scritti poemi ispirati al gioco,
come \textit{Scacchia ludus} di Marco Girolamo Vida.

Nell'ultimo secolo e mezzo, gli scacchi sono stati al centro di un intenso studio,
spesso guidato da personalità di spicco, come Bobby Fischer e Garry
Kasparov. Essi - insieme a moltissimi altri professionisti di alto
livello - hanno affrontato lo
sport con metodologia sistematica e strumenti matematici.

In fine, nel recentissimo periodo, tantissimi giovani hanno deciso di
approcciare l'ambito, forti della possibilità di
giocare online dal proprio smartphone, seguendo le 
personalità di spicco - come l'ex campione del Magnus Carlsen - 
anche sui social network. In particolare tali mezzi hanno permesso una
distribuzione capillare del gioco e delle figure di spicco, che
possono essere considerate, oltre che campioni e Grandi Maestri, anche 
influencer e celebrità.

\chapter{Requirements}

La presenza di un'ontologia che riporti le principali figure ed eventi
ufficiali (e non) dell'ambito, certamente sarebbe una fonte importante
per gli scopi didattici, nello specifico per conoscere ed imparare la
storia degli scacchi, ma anche nell'ambito divulgativo. Sarebbe
possibile, infatti, avere una base di conoscenza da cui attingere per
reperire informazioni su: partite giocate, stili di gioco, evoluzione
dei professionisti.
L'ontologia permette inoltre di fornire una base di
comunicazione fra diversi sistemi automatici che operano nel dominio degli
scacchi, come possono essere database, interfacce web, e siti per la
diffusione di informazioni, come le testate giornalistiche del settore.

\section{Finalità dell'ontologia}

Le finalità dell'ontologia costruita sono principalmente
quella \textbf{educativa} e \textbf{descrittiva}, oltre al poter dare
una formulazione più strutturata ai dati esistenti dell'ambito.
Fin dall'inizio dell'era corrente degli scacchi - ovvero da quando
sono entrate in vigore le regole correnti, nel finire del XIX secolo -
è sempre stato preferibile avere una descrizione testuale delle partite,
registrato tramite diversi stili di notazione. Tutte le partite
documentate in tale modo, dunque, rappresentano un'importante quantità di
dati, da cui è possbile estrarre informazioni interessanti. 
Elementi di interesse estraibili da tali dati potebbero riguardare la strategia di
gioco dei diversi scacchisti, la loro propensione al rischio, l'aggressività
delle mosse giocate. Avere un'ontologia strutturata per classi,
permette di avere una visione più chiara e ordinata di tali dati.

\section{Task dell'ontologia}

Fra i task principali dell'ontologia troviamo quello
\textbf{descrittivo}: rappresentare il dominio scacchistico in modo
formale e coerente permette di avere una comprensione più chiara dei
vari componenti che compongono tale dominio, e di come sono
relazionati fra loro. Permette inoltre di fornire una base di
comunicazione fra diversi sistemi che operano nel dominio degli
scacchi: tale ontologia può infatti fornire un vocabolario comune con
cui riferirsi a specifici concetti o individui. Un'ultima principale
motivazione per costruire tale ontologia sarebbe espandere ed
integrare la 
\href{https://www.bbc.co.uk/ontologies/sport-ontology/}{Sport Ontology} costruita da \textit{BBC}, per includere anche
lo scacchi, quale sport olimpico.


\section{Utenti dell'ontologia}

L'utenza a cui è rivolta l'ontologia è principalmente un pubblico di
studenti e appassionati di scacchi, che vogliono avere una visione
complessiva e strutturata del dominio. Gli elementi riportati nell'ontologia,
infatti, sono generalmente noti a chiunque abbia una conoscenza
approfondita dell'ambiente. Ciononostante, l'ontologia può essere
facilmente espansa per contenere elementi e classi più specifici,
potendo dunque contenere informazioni molto specifiche riguardo
elementi quali strategie giocate in determinate partite, o stili di
gioco delle principali personaggi dell'ambito.

\chapter{Descrizione del dominio}

Gli scacchi sono un gioco nato in India, il cui corrente regolamento 
fu definito attorno al 1880. 

È uno dei giochi da tavolo più popolari al mondo,
e permette di essere interpretato in modo agonistico,
grazie alla presenza di un organo regolatore, la FIDE.

Come molti altri domini, ha subito un'importante evoluzione con l'avvento del web e
della tecnologia moderna.

\begin{figure}[h]
  \caption{Scacchiera moderna dotata di sensori per la registrazione
  delle mosse. Lo strumento è dotato di una porta USB-C che permette
  la connessione di dispositivi di memoria o laptop con software
  relativo.}
  \centering
  \label{fig:scacchiera}
  \includegraphics[scale=1]{~/Pictures/Screenshots/Screenshot_20231231_141455.png}
\end{figure} 
Un primo elemento che ha permesso di
semplificare molto l'aspetto documentale
delle partite giocate fisicamente, sono state le scacchiere moderne,
come quella in figura \ref{fig:scacchiera} che
permettono la registrazione e la trasmissione delle mosse giocate,
generando dunque un discreto quantitativo di dati che è necessario
raccogliere, strutturare e rendere disponibile.

In aggiunta a ciò, la diffusione di internet ha permesso di giocare
partite
online, le quali contengono altrettanta informazione interessante,
specie se giocate fra grandi professionisti. 

Il recente picco di interesse che lo scacchi ha avuto, è anche grazie alla 
diffusione che ha subito nella cultura di
massa. Una menzione speciale va fatta a \textit{The Queen's Gambit}, 
serie televisiva Netflix del 2020, che ha avuto un impatto notevole nello
\textit{svecchiare} l'immagine dello sport.


Alcuni riferimenti importanti sono riportati nella seguente
sitografia:
\begin{itemize}
  \item
    https://digitalgametechnology.com/products/home-use-e-boards, è
    l'azienda produttrice delle scacchiere ufficiali per i tornei
    mondiali di scacchi, come quella sovrariportata.
  \item https://www.chess.com/, è il sito più famoso per giocare
    partite online, e contiene un'importante quantità di dati
    riguardanti le partite giocate. Ogni giocatore su questo sito deve
    dotarsi di un account, per cui è possibile interagire e osservare
    le partite che gli scacchisti professionisti giocano, oltre al
    poter sfidare i propri amici e cari.
  \item https://www.fide.com/ è il sito ufficiale della Federazione
    Internazione degli Scacchi. Su tale risorsa è possibile trovare
    informazioni riguardo i vari giocatori, i tornei, le regole, e le
    news più recenti riguardanti il dominio.
\end{itemize}

Per quanto riguarda i riferimenti bibligrafici, sono stati scritti
tantissimi testi riguardanti le varie fasi di gioco, e sul come vadano
affrontate, ma le informazioni di tali testi non sono stati utilizzati
nella presente relazione o nella costruzione dell'ontologia.


\chapter{Competency Questions}

\begin{enumerate}
  \item Quali sono le principali strategie scacchistiche?
  \item Quante partite ha vinto il giocatore [X]?
  \item Chi è stato l'ultimo vincitore del torneo [X]?
  \item Quale giocatore ha vinto più partite usando la strategia [X]?
\end{enumerate}

\chapter{Documentazione}

- regole del gioco estratte dal sito della fide (citazione necessaria)
- personaggi di spicco (presente e passato, per quelli presenti
rimandare ad elementi che mostrino come siano influencer su ig)
- Libri su aperture e strategie
- quale screenshot di chess.com


\chapter{LODE}

\chapter{Visualizzazione}

\end{document}
