\documentclass[12pt]{book}
\usepackage[italian]{babel}
\usepackage{amsfonts}
\usepackage{amsmath}
\usepackage{tcolorbox}
\title{Modellazione Concettuale per il Web Semantico \\ A.A. 2023-2024}
\author{Enrico Cassano \\ N.M. 912344}
\date{}

\usepackage{hyperref}

\begin{document}

\maketitle

\tableofcontents

\chapter{Motivazioni}

Gli scacchi sono un gioco nato in India nel VI secolo D.C, e sono
giunti in Europa attorno all'anno 1000, grazie agli influssi arabi
presenti nella penisola iberica. Raggiunsero la forma quasi attuale nel XV
secolo in Italia e in Spagna, ed il regolamento corrente fu definito
attorno al 1880. È uno dei giochi da tavolo più popolari al mondo,
grazie alla possibilità di essere giocato pressoché ovunque. Permette,
inoltre, di essere interpretato in modo agonistico,
grazie alla presenza di un organo regolatore, la FIDE.

La presenza di un'ontologia che riporti le principali figure ed eventi
ufficiali (e non) dell'ambito, certamente sarebbe una fonte importante
per gli scopi didattici, nello specifico per conoscere ed imparare la
storia degli scacchi, ma anche per l'ambito divulgativo. Sarebbe
possibile, infatti, avere una base di conoscenza da cui attingere per
reperire informazioni su: partite giocate, stili di gioco, evoluzione
dei professionisti.
L'ontologia permette inoltre di fornire una base di
comunicazione fra diversi sistemi automatici che operano nel dominio degli
scacchi, come possono essere database, interfacce web, e siti per la
divulgazione informazioni come le testate giornalistiche del settore.


\chapter{Requirements}

\section{Finalità dell'ontologia}

Le finalità principali dell'ontologia costruita sono principalmente
quello \textbf{educativo} e \textbf{descrittivo}, ma anche per dare
una formulazione pià strutturata ai dati presenti riguardo le partite.
Fin dall'inizio dell'era corrente degli scacchi - ovvero da quando
sono entrate in vigore le regole correnti, nel finire del XIX secolo -
è sempre stato preferito avere una descrizione testuale delle partite,
fatto anche tramite diversi stili di notazione. Tutte le partite
documentate in tale modo, dunque, rappresentano un'enorme quantità di
dati, da cui è possbile estrarre informazioni riguardo la strategia di
gioco dei partecipanti, la loro propensione al rischio, l'aggressività
delle mosse giocate. Avere un'ontologia strutturata per classi,
permette di avere una visione più chiara e ordinata di tali dati.

\section{Task dell'ontologia}

Fra i task principali dell'ontologia troviamo quello
\textbf{descrittivo}: rappresentare il dominio scacchistico in modo
formale e coerente permette di avere una comprensione più chiara dei
vari componenti che compongono tale dominio, e di come sono
relazionati fra loro. Permette inoltre di fornire una base di
comunicazione fra diversi sistemi che operano nel dominio degli
scacchi: tale ontologia può infatti fornire un vocabolario comune con
cui riferirsi a specifici concetti o individui. Un'ultima principale
motivazione per costruire tale ontologia sarebbe espandere ed
integrare la 
\href{https://www.bbc.co.uk/ontologies/sport-ontology/}{Sport Ontology} costruita da \textit{BBC}, per includere anche
lo scacchi, quale sport olimpico.


\section{Utenti dell'ontologia}

L'utenza a cui è rivolta l'ontologia è principalmente un pubblico di
studenti e appassionati di scacchi, che vogliono avere una visione
complessiva e strutturata del dominio. Gli elementi riportati nell'ontologia,
infatti, sono generalmente noti a chiunque abbia una conoscenza
approfondita dell'ambiente. Ciononostante, l'ontologia può essere
facilmente espansa per contenere elementi e classi più specifici,
potendo dunque contenere informazioni molto specifiche riguardo
elementi quali strategie giocate in determinate partite, o stili di
gioco delle principali personaggi dell'ambito.

\chapter{Descrizione del dominio}

\chapter{Competency Questions}

\chapter{Documentazione}

\chapter{LODE}

\chapter{Visualizzazione}

\end{document}
